% Careful: to be compiled with LuaLatex
\documentclass[10pt]{article}
\usepackage[usenames]{color} %used for font color
\usepackage{amssymb} %maths
\usepackage{amsmath} %maths
\usepackage[utf8]{inputenc} %useful to type directly diacritic characters
\usepackage{tikz-feynman}
\tikzfeynmanset{compat=1.0.0}

\begin{document}

% For Romain's thesis:
% g+g crossed to s+sbar
\feynmandiagram [layered layout, vertical=b to d] {
% Draw the top and bottom lines
i1 [particle=\(g\)]
-- [draw=none] b [dot]
-- [fermion] f1 [particle=\(s\)],
i2 [particle=\(g\)]
-- [draw=none] d [dot]
-- [anti fermion] f2 [particle=\(\overline{s}\)],
i1 -- [gluon] d,
i2 -- [gluon] b,
% Draw the two internal fermion lines
{ [same layer] b -- [anti fermion, edge label=\( \)] d},
{ [same layer] i1 -- [draw=none] i2},
};

% g+g square to s+sbar
\feynmandiagram [layered layout, vertical=b to d] {
% Draw the top and bottom lines
i1 [particle=\(g\)]
-- [gluon] b [dot]
-- [fermion] f1 [particle=\(s\)],
i2 [particle=\(g\)]
-- [gluon] d [dot]
-- [anti fermion] f2 [particle=\(\overline{s}\)],
% Draw the two internal fermion lines
{ [same layer] b -- [anti fermion, edge label=\( \)] d},
};

% g+g to gluon  to s+sbar
\feynmandiagram [baseline=(b.base), horizontal=a to b] {
i1 [particle=\(g\)] -- [gluon] a [dot] -- [gluon] i2 [particle=\(g\)],
a -- [gluon, edge label=\( \) %, momentum'=\(k\)
] b [dot],
f1 [particle=\(s\)] -- [anti fermion] b -- [anti fermion] f2 [particle=\(\overline{s}\)],
};


% q+qbar to gluon  to s+sbar
\feynmandiagram [baseline=(a.base), horizontal=a to b] {
i1 [particle=\(\overline{q}\)] -- [anti fermion] a [dot] -- [anti fermion] i2 [particle=\(q\)],
a  -- [gluon, edge label=\( \) %, momentum'=\(k\)
] b [dot],
f1 [particle=\(s\)] -- [anti fermion] b  -- [anti fermion] f2 [particle=\(\overline{s}\)],
};

\end{document}
