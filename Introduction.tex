%//------ Section 00 -------------------------------------------------------------------------------------------------
\chapter{Preface}
\label{chap:Chapter1}
%//-----------------------------------------------------------------------//

Nature is governed by a very few number of fundamental interactions: the gravitational, electromagnetic, strong and weak interactions. The comprehension of these forces was at the heart of research in Physics throughout the \upperRomannumeral{19}$^{\textrm{th}}$ and \upperRomannumeral{20}$^{\textrm{th}}$ centuries. This endeavor led to the two pillars of modern physics: Einstein's theory of general relativity, in which gravity is a geometric effect of the topology -- in particular, the curvature -- of spacetime, and the Standard Model of particle physics. In the latter case, the three other forces are understood as an exchange of elementary particles (vector gauge bosons or quanta) of their underlying quantum field.

Within the framework of the Standard Model, the strong interaction is described by quantum chromodynamics (QCD). In this theory, the \textit{quarks} --- the elementary particles sensitive to this force --- carry a \textit{colour} charge\footnote{This is the analog of the electric charge in QCD.}, that allows the exchange of \textit{gluons}, the vector gauge bosons of QCD. The pecularity of this theory resides in its non-Abelian structure, meaning that gluons themselves are colour-charged and thereby can self-interact. The direct consequence of this feature is the running of the QCD coupling constant with the energy scale. In processes involving large momentum transfer (or at short length scale), the coupling constant weakens and the partons -- quarks and gluons -- can be viewed as free particles, leading to asymptotic freedom. Conversely, for lower momentum exchange (or at larger distance, typically of the order of the proton size), the coupling increases forcing partons to be confined inside composite objects, named hadrons, made of two or three valence quarks: the \textit{mesons} and \textit{baryons} respectively. In this regime, QCD calculations can only be achieved via non-perturbative approaches. One of these reveals another compelling feature: Lattice QCD (lQCD) predicts a phase transition from hadronic to partonic matter at extremely high temperature and/or densities; since the partons are deconfined and -- similarly to plasmas -- interact weakly, this state of matter is called the \textit{quark-gluon plasma} (QGP). It is believed to have existed in the primordial Universe, merely a few microseconds after the Big Bang, and could be present in the core of neutron stars. \\

This QGP is not only a concept, it is an experimental fact. Although the first studies date from the 1970's \cite{carruthersQuarkiumBizarreFermi1974}\cite{harringtonHighDensityPhaseTransitions1974}\cite{collinsSuperdenseMatterNeutrons1975}, research on the QGP took off in 2000 with the hint of its existence by the experiments of the CERN (European Organisation for Nuclear Research) heavy ion programme \cite{NewStateMatter2023}. This was validated later, in 2005, by the experiments at the Relativistic Heavy Ion Collider (Brookhaven National Laboratory) \cite{ludlamHUNTINGQUARKGLUON2005}\cite{arseneQuarkGluonPlasma2005}\cite{alPHOBOSPerspectiveDiscoveries2005}\cite{phenixcollaborationFormationDensePartonic2005} \cite{starcollaborationExperimentalTheoreticalChallenges2005}.

Experimentally, the QGP is recreated in laboratory by colliding heavy nuclei (C, O, Au, Pb,...) at extremely high energies. Due to its fleeting existence of about $10^{-23} s$, the study of this exotic state of matter relies primarly on the observation of the footprints/signatures it left during the collision. The exploration of the QGP also hinges on more elemental collisions, namely proton-nucleus and proton-proton (pp) collisions, where no QGP is foreseen and which are therefore used as baselines. 

Among the various available probes of the QGP, the multi-strange baryons, \rmXi and \rmOmega containing two or three \textit{strange} quarks, appear as the preferred research approach. Being both light and heavy particles, they constitute exotic hadrons abundantly produced in the collision, that provide effective constrains on statistical models. Furthermore, thanks to a characteristic decay topology (cascade), their identification is possible on a vast domain of transverse momentum, originating from different production mechanisms, eventually intertwined. Finally, one key signature of the QGP is the \textit{strangeness enhancement}, which consists in the increased yields of strange quarks and thus, in the final state, of strange hadrons. In particular, this enhancement intensifies for hadrons with the largest strangeness content, namely the \rmXi and \rmOmega.\\

Nowadays, the experiment at CERN devoted to studying QCD- and QGP-physics is \textit{A Large Ion Collider Experiment} (ALICE), installed on the ring of the \textit{Large Hadron Collider} (LHC). After two campaigns of data taking in 2009-2013 (Run-1) and 2015-2018 (Run-2), the LHC accelerator has restarted on the 5th of July 2022 for a four-year programme (Run-3) \cite{ThirdRunLarge2023}. During the second long shutdown period of the collider (2018-2022), ALICE has been fully revamped and comes out now as a brand-new experiment: new Inner Tracking System with reduced material budget; improved Time Projection Chamber; installation of a Muon Forward Tracker; upgraded detectors joined with a new Online-Offline software to enable continuous readout of Pb-Pb collisions to interaction rate up to 50 kHz \cite{alicecollaborationUpgradeALICEExperiment2014}. Thanks to these upgrades, the study of QCD- and QGP-physics at LHC enters into a new age, an era of "precision".

About precision, it is enlightening to wonder what it truly means; after all, no one performs unprecise measurements. In the present context (and in my humble opinion), this encompasses two aspects: on one hand, a thorough exploration/characterisation of the object of study with new observables or impossible measurements now at reach; on the other hand, accurate measurements no longer dominated by either the statistical or systematic uncertainties.
In this respect, looking back at the achievements from the previous rounds of data taking, namely LHC Run-1 and Run-2, they are -- to a certain extent -- plenty of measurements, especially in the light flavour sector. For instance, we can mention \cite{alicecollaborationCharacterizingInitialConditions2022}\cite{schotterMultidifferentialInvestigationStrangeness2023}\cite{schotterQCDLHC2022}.\\

This thesis proposes pursuing this precision endeavor on multi-strange baryons thanks to the excellent tracking and identification capabilities (at mid-rapidity) of the ALICE experiment in the LHC Run-2. The focus is on pp collisions at a centre-of-mass energy of \sqrtS = 13 \tev. During this three-year PhD spanning from 2020 to 2023, two analyses have been performed; each one will be appropriately introduced and detailed in a dedicated chapter.

The manuscript opens with an introduction of (modern) particle physics in \chap\ref{chap:ParticlePhysics}. The basic concepts of the Standard Model are presented, with a detailed description of the strong interaction. The notion of QGP is also explained, from its formation to its experimental signatures. One of these, the strangeness enhancement, receives a more particular attention.

It is followed by the \chap\ref{chap:ALICE}, the usual chapter that provides an overview of the ALICE collaboration. First, the direct surroundings of the ALICE experiment is depicted, that is the CERN, its accelerator complex and the main experiments installed on the ring of the LHC. Then, the internal structure of the collaboration will be presented, shortly accompanied by the showcase of the main sub-detectors of the ALICE experiment and particularly the ones used in the analyses reported in this manuscript. The event, vertex and tracks reconstruction procedures are mentionned. This chapter insists on the technique employed for identifying and selecting the characteristic cascade decay of the multi-strange baryons \rmXi and \rmOmega. The choice of the ALICE experiment for studying those particles in the context of this thesis is also justified.

The \chap\ref{chap:CPTAnalysis} gives the details on the first analysis of multi-strange baryons. It consists in measuring the \rmXiM, \rmAxiP, \rmOmegaM, \rmAomegaP  masses and mass differences between particle and anti-particle using pp collisions at \sqrtS = 13 \tev. The values of the latter offer the opportunity to test the validity of the CPT symmetry to an unprecedented level of precision in the multi-strange baryon sector. This chapter showcases the challenge and the difficulties that one faces with such a measurement.

A second analysis has been carried out based on the experience gained from the first one. It is detailed in \chap\ref{chap:CorrelatedAnalysis}. It aims at studying the correlated production of strange hadrons in order to shed more light on the origin of the strangeness enhancement in pp collisions. The physical interpretation of the results is based on the comparison of our measurement to various QCD-inspired Monte-Carlo models. The primary focus is to correlate a multi-strange baryon (\rmXi or \rmOmega) with a \rmPhiMes resonance ($s\bar{s}$), but other kind of correlations are also considered.


The final chapter, \chap\ref{chap:Conclusion}, consists in a discussion on the results of both analysis. Different extensions of the present work are proposed.