\chapter*{Acknowledgements}

This thesis is the culmination of a collaborative effort involving dozens of people, each of whom contributed more or less directly, with a major or minor role, knowingly or not, spontaneously or spread out throughout this three-year PhD. In this section, I wish  to thank each and every one of these contributors.\\

I would like first to start to express my deepest gratitude to Jérôme Baudot for presiding over the jury of this thesis, but most of all for introducing me to particle physics during my first-year master's internship in 2018. There are decisive moments in one's life, and this one certainly belongs to the top of the list. I would also like to warmly thank all the members of the jury for kindly accepting to judge this work and for being present: Johanna Stachel, Federico Antinori, Tanguy Pierog, and especially the referees, Raphaël De Cassagnac and Patrick Robbe. Thank you for the careful reading and for the fruitful discussions. \\

Je vous remercie Antonin et Boris pour m'avoir suivi et aiguillé tout au long de mon parcours. Depuis ce stage d'été en 2019, j'ai énormément appris à vos côtés. Votre compréhension de la physique, votre expérience, votre sagesse, votre pédagogie, votre enthousiasme, votre bonne humeur en toute circonstance\footnote{Mis à part lorsque l'on parle de la lourdeur administrative, ahah.}, votre disponibilité, votre soutien, ..., font de vous des modèles pour moi. Vous m'avez fait grandir aussi bien sur le plan individuel que scientifique. Je n'ai pas les mots pour exprimer l'entièreté de ma gratitude. \`A défaut : merci !\\

Cet ouvrage est également le fruit du travail d'une (excellente) équipe. J'aimerais donc remercier l'ensemble de l'équipe ALICE de l'IPHC pour m'avoir accueilli au sein de l'équipe dès la première année de master en 2019, pour sa bonne humeur, pour ses conseils avertis, pour ses réunions de groupe bien trop longues le mercredi matin. Mention spéciale à : Alexandre Bigot\footnote{A.k.a \textit{water thief}. Pour avoir été là toutes ces années, pour toutes nos questions et nos longs débats, et pour supporter le bruit agaçant de mon clavier, merci.}, Antonin Maire\footnote{Il y aurait tant de choses à dire ! Le hasard a fait que tu es venu avec moi lorsque j'avais oublié ma valise à l'hôtel du CERN en janvier 2019, et qu'une chose en entrainant une autre, on a discuté de réaliser un stage dans l'équipe ALICE à Strasbourg. Le début d'une grande aventure. Merci de m'avoir accompagné ce jour-là !}, Arthur Gal\footnote{Pour m'introduire aux us et coutumes de l'équipe, répondre à mes questions et éclaircir mes idées.}, Boris Hippolyte\footnote{Trop souvent le premier arrivé le matin. Toujours là pour répondre à mes mails à 2 h du mat', pour râler qu'il n'y a plus de papier/d'encre dans l'imprimante, pour boire un verre. C'est à toi que je dois mon nom sur la porte du bureau, le premier à m'avoir emmené en conférence, à m'avoir introduit à la communauté française de la physique des ions lourds, qui m'a donné le sentiment d'appartenir à une équipe après les multiples confinements, qui a préparé les cocktails le jour de ma soutenance... La liste est longue. A défaut d'être exhaustif, j'espère que ces quelques mots suffiront : un grand merci pour tout !}, Christian Kuhn\footnote{Merci pour toutes nos discussions diverses et variées ! Je n'ai pas oublié ma promesse : je m'engage solennellement à remporter, au moins, une partie d'échecs contre toi, ahah.}, Fouad Rami\footnote{Merci beaucoup pour tes questions de fond, pour les longues discussions qui en ont suivi et, par conséquent, pour les réunions d'équipe à n'en pas finir :-)}, Iouri Belikov\footnote{The team, and my experience as a PhD student,  would not be the same without you. Thank you so much for all our coffee discussions, answering my stupid questions, for the fascinating service task on the pre-alignment of the ITS-2, our chess games. Together, I am sure that one day, we will beat Christian.}, Marc Imhoff\footnote{Un frère Alsacien dans cette équipe de Lorrains ! Je chérirai toujours nos longues discussions et nos échanges de ragots.}, Sergei Senyukov\footnote{Connaissant toujours les bonnes adresses, les bons films, et pleins d'anecdotes et de détails techniques sur l'expérience ALICE.}, Yitao Wu\footnote{For his friendly attitude and his open-mindedness.}, Yongzhen Hou\footnote{The mysterious benefactor, always filling the fridge with food :-)}, Yves Schutz\footnote{Pour ton encadrement lors de mon stage volontaire durant l'été 2019, ton humeur, ton enthousiasme, ta disponibilité, ta réactivité, tes commentaires en profondeur et le suivi de mon évolution.}. Vous avez tous contribué à ce que ce travail se déroule dans d'excellentes conditions. Merci pour tous ces moments enrichissants et amusants. \say{Que c'est un merveilleux assaisonnement aux plaisirs qu'on goûte que la présence des gens qu'on aime}\footnote{\textit{Le Misanthrope}, Molière.}.\\

Je n'oublierai pas tous les étudiants qui sont passés dans l'équipe et avec qui j'ai partagé des moments mémorables : Alexandre Bigot, Antoine Grillet, Océane Poncet, Idriss Larbi, Gaël Coulon, Arthur Dedieu, Stanislas Lambert et Romain Astorga-Petit.\\

Ma reconnaissance va également à Alessia Romagnoli\footnote{Pour ta réactivité, ta disponibilité, ta pédagogie, ta bonne humeur, ta tolérance vis-à-vis de mes retards pour clôturer mes missions. Avec toi, cela devient presque un plaisir d'avoir à réaliser des démarches administratives.}, Josiane Heidmann\footnote{Toujours présente pour discuter, avec le sourire, pour donner un coup de main, pour faire les choses bien.} et Nicolas Busser\footnote{Pour ton attitude amicale, pour ton esprit vif, pour tes connaissances/conseils sur le laboratoire, pour les photos lors de la soutenance de thèse, pour m'avoir encouragé à rejoindre le Bureau Des Doctorants.}. Je tiens également à remercier l'ensemble du Bureau Des Doctorants de l'IPHC (Emma Monpribat, Nicolas Dari Bako, Elisa Le Roux, Pierre Bourdier, Jean Soudier, Gaël Simonin, Marie Gébelin, Jérôme Castel, Clément Parnet) avec qui j'ai entretenu d'excellentes relations. Mon passage dans le Bureau m'a permis d'en apprendre énormément sur comment créer une vie sociale au sein du laboratoire, et cela n'aurait pas été possible sans vous. Merci à tous !\\

Je ne saurais oublier tous ceux avec qui j'ai partagé mes activités d'enseignements~: Maaloum Mounir, Jérôme Combet, Mebarek Alouani, ainsi que Christian Boily et Fabrice Thalmann avec qui j'ai pris beaucoup de plaisir à travailler. Il est clair que ma mission d'enseignement n'aurait pas été la même sans vous. Pour tout ce que vous avez fait, pour votre présence, pour votre écoute au cours de ces trois années, merci beaucoup !\\ 

Je remercie également tous ceux qui m'ont formé et accompagné tout au long de ce chemin parcours : Jean Farago\footnote{\label{stageL3}Merci beaucoup pour nous avoir accepté en stage, Alexandre et moi, en 2018. Ce que l'on a appris au cours de cette première expérience en laboratoire transparaît encore aujourd'hui dans mon travail.}, Thierry Charitat\footnoteref{stageL3}$^{,}$\footnote{\label{FPT}Pour son aide précieuse et son humour qui a fait du \textit{French Physicist Tournament} une expérience inoubliable.}$^{,}$\footnote{Pour avoir surveillé mon évolution.}, Pierre M\"uller\footnoteref{FPT}, \'Eric Chabert\footnote{Pour sa bonne humeur, son esprit curieux, et son cours d'option en M1 qui m'a introduit à la physique des particules.}, Mathieu Goffe\footnote{Pour m'avoir supporté tout au long de l'EX2 sur SiTrInEO en janvier 2020.}, Hervé Molique\footnote{Pour ta pédagogie, pour lever le voile de mystère entourant la mécanique quantique, et pour consolider mes connaissances de la Physique.}, Janos Polonyi\footnote{Pour sa sagesse, sa gentillesse, sa patience pour expliquer la mécanique quantique relativiste et la théorie quantique des champs.}, Michel Rausch De Traubenberg\footnote{Pour les excellents cours de relativité restreinte et générale.}.\\

A part of this work has also been carried out in close collaboration with David Dobrigkeit Chinellato\footnote{For providing the analysis tasks and the pre-processed data. My understanding of the ALICE framework and the weak decay reconstruction, I owe it to you. Without a doubt, you stand as one of my PhD supervisors. Thank you!}, Kai Schweda\footnote{The first one to propose to measure the masses and mass differences between particle and anti-particle for \rmXi and \rmOmega particles. It was way more difficult than expected, but we did it. It was nice journey, and I owe it to you. Next time, drinks are on me!}, and Georgijs Skorodumovs\footnote{For initiating a different measurement, a \pT-differential measurement, of the mass of particles.}. Also, thank you, Anders Garitt Knospe, for providing the files for reconstructing \rmPhiMes resonance decays, thus laying the foundations of the correlated production analysis. \\

I would also like to thank all the members of the PWG-LF and PAG-Strangeness, and particularly all the convenors from 2020 to 2023 (Anders Garritt Knospe, Livio Bianchi, Roman Lietava, Lee Barnbee, Marek Bombara, Chiara De Martin, Ramona Lea, Alberto Caliva, Nicolo Jacazio), for following the progress of my analyses and for enduring my often far too long presentations. I am also grateful to Francesco Mazzaschi and Livio Bianchi for being part of the Analysis Review Committee on the CPT analysis. Your comments, and the following discussions, were a great help to shape the final results.\\

Je tiens également à remercier Brigitte Cheynis pour m'avoir encouragé à devenir ambassadeur des \textit{juniors} français de ALICE, et Sizar Aziz pour avoir passé le flambeau et sans qui je n'aurais pas pu vivre cette expérience. Je remercie également l'ensemble des \textit{juniors} français avec qui j'ai passé des moments inoubliables et qui ont su rendre le rôle d'ambassadeur aisé.\\

This work of the Interdisciplinary Thematic Institute QMat, as part of the ITI 2021 2028 program of the University of Strasbourg, CNRS and Inserm, has been supported by IdEx Unistra (ANR 10 IDEX 0002), and by SFRI STRAT’US project (ANR 20 SFRI 0012) and EUR QMAT ANR-17-EURE-0024 under the framework of the French Investments for the Future Program. In other words, thank you for financing my thesis!\\

Je remercie particulièrement ma famille, pour son soutien constant tout au long de mes études, ainsi que mes amis pour leur amitié qui m'est chère : Alexandre Bigot, Anne Rieb, Anthony Leduc, Antoine Grillet, Emma Monpribat, Fernando Flor, Florian Schotter, Lucas Martel, Maxime Grillet, Margaux Forge, Mario Sessini, Raphael H\"aberle, Valentin Goetz, Victor Heilmann, Vincent Juste, et tous ceux que j'ai pu oublier. \\

Mes derniers remerciements, je tiens à les adresser à celle qui est au centre de ma vie, Camille Bottemer. Tu m'as encouragé, tu m'as supporté -- dans les deux sens du terme -- tout au long de mon parcours ; tu as su faire oublier les moments difficiles et donner la force nécessaire pour les affronter. Du plus profond de mon c\oe{}ur, merci d'avoir été présente, encourageante et compréhensive. Ça a été un privilège de t'avoir à mes côtés, et j'espère faire face à encore beaucoup d'épreuves avec toi.
