%//------ Section 03 -------------------------------------------------------------------------------------------------
\chapter{Complementary materials for the analysis:\\Analysis of the correlated production of strange hadrons}
\label{appendix:CorrelatedAnalysis}
%//-----------------------------------------------------------------------//

\section{\Epos configuration}
\begin{verbatim}
set laproj 1 !projectile atomic number
set maproj 1 !projectile mass number
set latarg 1 !target atomic number
set matarg 1 !target mass number
set ecms 13000 !sqrt(s)_pp
set istmax 25 !max status considered for storage 
set iranphi 1 !for internal use. if iranphi=1 event will be rotated such that 
              !the impact parameter angle and the (n=2) event plane angle
              !(based on string segments) coincide. Particles rotated back at the end. 
ftime on     !string formation time non-zero
!suppressed decays: 
nodecays 
 110 20 2130 -2130 2230 -2230 1130 -1130 1330 -1330 2330 -2330 3331 -3331 
end

set ihepmc 1
set ninicon 1            !number of initial conditions used for hydro evolution
core off                 !core/corona not activated
hydro off                !hydro not activated
eos off                  !eos not activated
hacas off                !hadronic cascade not activated  
set nfreeze 1            !number of freeze out events per hydro event 
set modsho 1             !printout every modsho events
set centrality 0         !0=min bias 

!fillTree(C1)               !uncomment to get root tree output
\end{verbatim}


\section{\Pythiaeight configuration with colour reconnection enabled}
\begin{verbatim}
# Parameter of the MPI model to keep total multiplicity reasonable
MultiPartonInteractions:pT0Ref = 2.15

# Parameters related to Junction formation/QCD based CR
BeamRemnants:remnantMode = 1
BeamRemnants:saturation = 5
ColourReconnection:mode = 1
ColourReconnection:allowDoubleJunRem = off
ColourReconnection:m0 = 0.3
ColourReconnection:allowJunctions = on
ColourReconnection:junctionCorrection = 1.2
ColourReconnection:timeDilationMode = 2
ColourReconnection:timeDilationPar = 0.18

# Enable rope hadronization
Ropewalk:RopeHadronization = on

# Also enable string shoving, but don't actually do anything.
# This is just to allow strings to free stream until hadronization
# where the overlaps between strings are calculated.
Ropewalk:doShoving = on
Ropewalk:tInit = 1.5 # Propagation time
Ropewalk:deltat = 0.05
Ropewalk:tShove = 0.1
Ropewalk:gAmplitude = 0. # Set shoving strength to 0 explicitly

# Do the ropes.
Ropewalk:doFlavour = on

# Parameters of the rope model
Ropewalk:r0 = 0.5 # in units of fm
Ropewalk:m0 = 0.2 # in units of GeV
Ropewalk:beta = 0.1

# Enabling setting of vertex information is necessary
# to calculate string overlaps.
PartonVertex:setVertex = on
PartonVertex:protonRadius = 0.7
PartonVertex:emissionWidth = 0.1
\end{verbatim}


