%//------ Section 07 -------------------------------------------------------------------------------------------------
\chapter{Discussion and conclusion}
\label{chap:Conclusion}
%//-----------------------------------------------------------------------//

At the beginning of this three-year PhD, in 2020, the LHC was in the middle of its second long shutdown. For an experiment installed on the collider -- such as \mbox{ALICE} --, this is a decisive moment: the experiment has to carry out its major upgrade programme on time for the start of the LHC Run-3 and, simultaneously, it must finalise the physics analyses of the previous data taking period. The present thesis contributes to the latter. In particular, it proposes to push the data analysis to the limits of ALICE during the LHC Run-2, by performing precision measurements in the light flavour sector, with multi-strange baryons. In that regard, two analyses have been performed.\\


\begin{table}[h]
    \hspace{-1.3cm}
    \begin{tabular}{cccc|ccc}

%    \begin{tabular}{b{2cm}@{\hspace{0.5cm}} b{3cm}@{\hspace{0.5cm}} b{2cm}@{\hspace{0.5cm}} b{2cm}@{\hspace{0.5cm}} b{5cm}@{\hspace{0.5cm}} b{3cm}@{\hspace{0.5cm}} b{3cm}@{\hspace{0.5cm}}}
    \noalign{\smallskip}\hline \noalign{\smallskip}
    \bf Particle & \bf Measured & \multicolumn{2}{c|}{\bf Uncertainty} & \bf Previous & \multicolumn{2}{c}{\bf Uncertainty}\\
    & \bf mass & \bf stat. & \bf syst. & \bf measured mass & \bf stat. & \bf syst.\\
    & (\mmass) & (\mmass) & (\mmass) & (\mass) & (\mmass) & (\mass) \\
    \noalign{\smallskip}\hline \noalign{\smallskip}
    \rmXiM & 1321.968 & 0.025 & 0.070 & 1321.70 & 0.08 & 0.05 \\
	\rmAxiP & 1321.918 & 0.025 & 0.075 & 1321.73 & 0.08 & 0.05 \\
    \noalign{\smallskip}\hline \noalign{\smallskip}
    \rmOmegaM & 1672.520 & 0.033 & 0.102 & 1673 & \multicolumn{2}{c}{1} \\ 
    \rmAomegaP & 1672.571 & 0.033 & 0.101 & 1673 & \multicolumn{2}{c}{1} \\ 
	\noalign{\smallskip}\hline \noalign{\smallskip}
	\bf Particle & \bf Measured & \multicolumn{2}{c|}{\bf Uncertainty} & \bf Previous & \multicolumn{2}{c}{\bf Total}\\
    & \bf mass difference & \bf stat. & \bf syst. & \bf mass difference & \multicolumn{2}{c}{\bf uncertainty} \\
    & ($\times 10^{-5}$) & ($\times 10^{-5}$) & ($\times 10^{-5}$) & ($\times 10^{-5}$) & \multicolumn{2}{c}{($\times 10^{-5}$)}\\
    \noalign{\smallskip}\hline \noalign{\smallskip}
    \rmXi & -3.34 & 2.67 & 6.61 & 2.5 & \multicolumn{2}{c}{8.7} \\
    \noalign{\smallskip}\hline \noalign{\smallskip}
    \rmOmega & 3.44 & 3.00 & 2.51 & 1.44 & \multicolumn{2}{c}{7.98}\\ 
	\noalign{\smallskip}\hline \noalign{\smallskip}
    \end{tabular}
    \caption{On the left: final measured masses and relative mass differences for \rmXiPM and \rmOmegaPM, with their associated statistical and systematic uncertainties. On the right: previous measurements of the mass and relative mass difference for the \rmXiPM \cite{abdallahMassesLifetimesProduction2006} and \rmOmegaPM \cite{chanMeasurementPropertiesOverline1998, hartouniNclusiveRoductionEnsuremath1985}, with their statistical and systematic uncertainties. If the latter are not quoted in the paper, the total uncertainty is indicated.}\label{tab:FinalResultsCPT}
\end{table}
%\begin{table}[h]
%    \centering
%    \begin{tabular}{cccc|ccc}
%
%%    \begin{tabular}{b{2cm}@{\hspace{0.5cm}} b{3cm}@{\hspace{0.5cm}} b{2cm}@{\hspace{0.5cm}} b{2cm}@{\hspace{0.5cm}} b{5cm}@{\hspace{0.5cm}} b{3cm}@{\hspace{0.5cm}} b{3cm}@{\hspace{0.5cm}}}
%    \noalign{\smallskip}\hline \noalign{\smallskip}
%    \bf Particle & \bf Measured & \multicolumn{2}{c|}{\bf Uncertainty} & \bf Measured & \multicolumn{2}{c}{\bf Uncertainty}\\
%    & \bf mass & \bf stat. & \bf syst. & \bf mass difference & \bf stat. & \bf syst.\\
%    & (\mmass) & (\mmass) & (\mmass) & ($\times 10^{-5}$) & ($\times 10^{-5}$) & ($\times 10^{-5}$) \\
%    \noalign{\smallskip}\hline \noalign{\smallskip}
%    \rmXiM & 1321.968 & 0.025 & 0.070 & \multirow{2}{*}{-3.34} & \multirow{2}{*}{2.67} & \multirow{2}{*}{6.61} \\
%	\rmAxiP & 1321.918 & 0.025 & 0.075 & & & \\
%    \noalign{\smallskip}\hline \noalign{\smallskip}
%    \rmOmegaM & 1672.520 & 0.033 & 0.102 & \multirow{2}{*}{3.44} & \multirow{2}{*}{3.00} & \multirow{2}{*}{1.91} \\ 
%    \rmAomegaP &  1672.571 & 0.033 & 0.101 & & & \\ 
%	\noalign{\smallskip}\hline \noalign{\smallskip}
%    \end{tabular}
%    \caption{Final measured masses and relative mass differences for \rmXi and \rmOmega, with their associated statistical and systematic uncertainties.}\label{tab:FinalResultsCPT}
%\end{table}

The first analysis of this thesis consists in a precise measurement of the \rmXiM, \rmAxiP, \rmOmegaM, \rmAomegaP masses and mass differences between particle and antiparticle. The main motivation is that the last mass measurements of such nature have been performed 17 and 25 years ago, and rely on a low statistics. In contrast, the present analysis makes use of the excellent reconstruction capabilities of ALICE during the\break LHC~Run-2, and the abundant production of strange hadrons in pp collisions at a centre-of-mass energy of 13 \tev: about 2 500 000 $(\rmXiM + \rmAxiP)$ and approximately 133~000~$(\rmOmegaM + \rmAomegaP)$.  

Through the \chap\ref{chap:CPTAnalysis}, it has been shown that a fine comprehension of the data reconstruction is required to perform such measurements, and quickly the limits of the detector calibration are reached. To overcome these limitations, a sizeable fraction of the statistics had to be sacrificed. The final measurements -- summarised in \tab\ref{tab:FinalResultsCPT} -- can still compete with the latest measurements listed in the PDG, and improves them by a factor 1.20 and 2 for the relative mass difference of the \rmXi~and~\rmOmega baryons respectively (\fig\ref{fig:MassDiffVsPDGFinal}). Considering their precision, both are compatible with the CPT invariance symmetry. The presented results are in their final state, and should lead to a publication in the future. An Analysis Review Committee has been formed; a first version of an ALICE analysis note has already been reviewed.\\

\begin{figure}[h]
%\centering
\hspace*{-2cm}
\subfigure[]{
	\includegraphics[width=0.6\textwidth]{Figs/Chapter5/MassDiff\_Xi\_New.eps}
	\label{fig:MassDiffXiVsPDGFinal}
} 
\subfigure[]{
	\includegraphics[width=0.6\textwidth]{Figs/Chapter5/MassDiff\_Omega\_New.eps}
	\label{fig:MassDiffOmegaVsPDGFinal}
}
\caption{Comparison of our mass difference values between the \rmXiM and \rmAxiP (a), and the \rmOmegaM and \rmAomegaP, to the past measurements quoted in the PDG, as of 2023 \cite{particledatagroupReviewParticlePhysics2022}. The vertical line and the shaded area represent the PDG value and its associated uncertainty.}
	\label{fig:MassDiffVsPDGFinal}
\end{figure}

Based on the first analysis, this thesis work continues with a second one whose objective is to extend the study of multi-strange baryons to their production mechanisms  in proton-proton collisions at the LHC energies. In particular, we want to understand how strangeness distributes in the event, and ideally trace the flux of strange quarks. To that end, the idea is to correlate particles: either strange to non strange particles ($\rmLambdaPM - \proton$, $\rmLambdaPM - \rmPiPM$, etc), or among strange hadrons. The latter encompasses different kind of correlations:  baryon to meson ($\Lambda [uds] - \rmKzeroS [d\bar{s}]$, etc) or baryon to baryon ($\rmXiM [dss] - \rmAlambda [\bar{u}\bar{d}\bar{s}]$, $\rmXiM[dss] - \rmAxiP [\bar{d}\bar{s}\bar{s}]$, $\rmOmegaM[sss] - \rmAomegaP [\bar{s}\bar{s}\bar{s}]$, etc). In any case, this requires measuring correlations between two \emph{identified} particles. 

Among the possibilities, we have implemented an analysis flow for studying multi-strange baryon -- either a \rmXiPM or a \rmOmegaPM\ -- to \pOrPbar, \rmPiPM, \rmKPM, \rmKstarZero, \rmKzeroS, \rmLambdaPM, \rmXiPM, \rmOmegaPM correlations. In practice, the analysis concentrates specifically on correlating a multi-strange baryon to a \rmPhiMes resonance. Such a measurement turns out to be rather challenging: the goal is to correlate two \textit{identified} particles, with a relatively low production rate\footnote{Over a thousand event, one can expect approximately 38 \rmPhiMes, 20 \rmXi and 2 \rmOmega at mid-rapidity ($\absrap < 0.5$) in pp collisions at \sqrtS = 13 \tev \cite{alicecollaborationProductionLightflavorHadrons2021}. In addition, they must belong to the same event in order to be useful in the analysis.}. In contrast, while the first analysis targets high purity, this one clearly aims for high efficiency.

This experimental constraint is important to distinguish between different phenomenological models. For instance, the Monash 2013 tune of \Pythia predicts an enhancement of the \rmOmega abundancy in presence of a \rmPhiMes in the event that decreases with the charged particle multiplicity, while the colour reconnection and colour rope \say{tune} anticipates an increase. 

%Preliminary results related to such correlation are presented in \chap\ref{chap:CorrelatedAnalysis}. They indicate that the available statistics for both \rmOmega baryons and \rmPhi resonances remains too low in minimum-bias proton-proton collisions; concerning the \rmXi hyperons, the angular and rapidity correlations have been studied separately; no structure in the rapidity-dependent correlations can be observed at the moment, while a strong azimuthal correlation arises. To gain more insights on the mechanisms at stake, a comparison between our experimental measurement and different MC model predictions would be required. This aspect has already been initiated -- mainly with \Pythia and \Epos in \appdx\ref{appendix:CorrelatedAnalysis} --, as well as an extension of the analysis towards high-multiplicity proton-proton collisions. These do not appear in the manuscript, as the results are not mature enough to draw any conclusion.\\

Preliminary results related to such correlation are presented in \chap\ref{chap:CorrelatedAnalysis}. They indicate that the available statistics of both \rmOmega baryons and \rmPhi resonances remains too low in minimum-bias proton-proton collisions; concerning the \rmXi hyperons, the angular and rapidity correlations have been studied separately. No structure in the rapidity-dependent correlations can be observed at the moment, while a local azimuth correlation arises. To gain more insights on the mechanisms at stake, a comparison between our experimental measurement and different MC model predictions has been performed. This aspect has been carried out focusing mainly on \Pythia and \Epos (\appdx\ref{sec:QCDMCModels}). It indicates that the correlated production of \rmXiPM baryons and \rmPhiMes resonances in minimum-bias proton-proton collisions is likely an interplay between soft and hard hadronisation mechanisms. 

The analysis has also been extended towards high-multiplicity proton-proton collisions. The same trend as in minimum-bias data can be observed, although the correlation appears as less prominent suggesting that the \rmPhiMes production is also achieved via other mechanisms that does not involve the production of a \rmXiPM hyperon. Concerning the \rmOmegaPM-\rmPhiMes correlation, no dependence on the rapidity separation can be identified whereas one can be seen with the azimuth. However, due to statistical limitations, no definite conclusions can be drawn.\\

These two analyses put into perspective the limits of the ALICE detector during the LHC Run-2. On the one hand, as shown in \chap\ref{chap:CPTAnalysis}, the uncertainties on the mass and mass difference values are driven by the detector calibration. In particular, the dominant source of systematic bias comes from residual mis-calibrations. To keep them under control, a sizeable fraction of the data had to be discarded, resulting in higher uncertainties. On the other hand, the second analysis lacks of statistics, making it more difficult to draw any firm conclusions. The solution to these limitations may eventually be found in the LHC Run-3.

As mentioned in the introduction, the ALICE collaboration carried out a major upgrade of the experimental apparatus during the LHC second long shutdown (2018-2022) with two main objectives: improve the spatial resolution of the tracking system, and increase the data taking rates. Thanks to these upgrades, over the whole 2022 data collection period, the experiment has recorded about 35 \invpb of pp collisions at \sqrtS~=~13.6~\tev \cite{cern152ndLHCCMeeting2022}. As a comparison, the inspected luminosity over the whole LHC Run-2 for minimum-bias pp collisions amounts to 0.059 \invpb, and to 13 \invpb for high-multiplicity collisions. In other words, over a one-year period, the available statistics for minimum-bias pp events have been increased by a factor~300.

However, this achievement comes with a cost: to reach such data taking capabilities, some detectors have to be pushed to their limits leading to instabilities. In particular, at such interaction rates, the ions in the drift volume of the TPC start to accumulate as a space-charge, creating distortions in the drift field and thus deteriorating the tracking performance. This space-charge effect can be corrected by applying the appropriated calibration. The TPC being the main tracking device, the key of the ALICE data taking revolves around the control of the space-charge distortions. Therefore, as in the first analysis of the manuscript in \chap\ref{chap:CPTAnalysis}, the whole challenge is to derive an accurate calibration. Although the current TPC calibration is not nearly sufficient to improve the analysis using the LHC Run-3 data, ALICE has emerged from the second long shutdown as a brand-new experiment. The overall performances keep improving and hopefully, in the coming months, a potentially better calibration than in the LHC Run-2 will be available.

Moreover, the ITS has been replaced with a new Inner Tracking System, the \mbox{ITS-2}, with a better spatial resolution and a reduced material budget. In view of improving the overall calibration of the ALICE detector, a fraction of the thesis has been dedicated to the pre-alignment of the ITS-2 detector. This is a critical stage in the commissioning of the detector, as it acts as an input for the final alignment of the apparatus. The current global alignment of the ALICE detector have been performed at the end of 2022, using the pre-alignment parameters identified during this thesis work.

In operation in pp collisions during the 2022 and 2023 data taking periods, the ITS-2 detector proves to be robust with typically 98-99\% of ALPIDE sensors that are operational \cite{alicecollaborationALICEUpgradesLHC2023}. The high availability, coupled with the high detection efficiency per layer, limit drastically the losses in the acceptance of the detectors.\\

%Furthermore, in view of improving the overall calibration of the ALICE detector, a fraction of the thesis has been dedicated to the pre-alignment of the ITS-2 detector. This is critical stage in the commissionning of the detector, as it acts as input for the final alignment of the apparatus. The current global alignment of the ALICE detector have been performed at the end of 2022, using the pre-alignment parameters identified during this PhD.
%
%Moreover, the ITS has been replaced with a new Inner Tracking System, the ITS-2, with a better spatial resolution and a reduced material budget. In operation in pp collisions in 2022 and 2023, the ITS-2 detector proves to be robust with typically 98-99\% of ALPIDE sensors that are operational \cite{alicecollaborationALICEUpgradesLHC2023}. The high availability, coupled with the high detection efficiency per layer limit drastically the losses in the acceptance of the detectors.\\

Similarly to the analyses performed throughout this thesis work, the\break LHC Run-3 has its share of challenges. It offers an improved track reconstruction, a better calibrated detector and a prodigious amount of statistics. Considering the limitations highlighted in this thesis, the next precision measurements can only be achieved with this upgraded version of ALICE. Although there is still a long way to go, one thing is certain: the precision era is ahead.

%As mentioned in the introduction, the ALICE collaboration carried out a major upgrade of the experimental apparatus during the second long shutdown (2018-2022). The ITS has been replaced with a new Inner Tracking System, the ITS-2, with a better spatial resolution and a reduced material budget\footnote{To that regard, the present author contributed to the commissionning of the ITS-2, and particularly to the pre-alignement of the detector.}. As opposed to its predecessor in the LHC Run-1 and Run-2, all the layers are equipped with the same technology (Monolithic Active Pixel Sensors), thus limiting the holes in the detector acceptance. Furthermore, ALICE deployed a new Online-Offline software to enable continuous readout of Pb-Pb collisions to interaction rates up to 50 kHz. Therefore, the experiment is now able to record much more data than in the LHC Run-2\footnote{As a comparison, over the past six months, ALICE already recorded 300 times more data than in the LHC Run-2.}. 
%
%In turn, ALICE has become much more dependent on the quality of its calibration, and most particularly, the one in the TPC. However, this is also an opportunity, as it  opens the way towards a potentially better calibration in the LHC Run-2. 

\newpage
    

