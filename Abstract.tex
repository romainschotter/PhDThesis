\chapter*{Abstract}

Quantum chromodynamics (QCD) predicts the existence of an extreme state of nuclear matter in which quarks and gluons are deconfined and thermalised: this is the so-called \textit{Quark Gluon Plasma} (QGP). The QGP has been studied experimentally at colliders such as the LHC at CERN in Geneva, during the LHC Run-1 (2009-2013) and Run-2 (2015-2018) data taking periods. The 5$^{\rm th}$ of July 2022, the LHC has restarted for a third data taking campaign (LHC Run-3), as well as the experiment in which this thesis is carried out, ALICE. This thesis proposes to analyse -- possibly, one last time -- the data recorded during the LHC Run-2 before moving on to the ones from the LHC Run-3, in order to fully exploit them and push them to their precision limits. To that end, two analyses have been performed.

The main analysis consists in a test of the CPT (Charge-Parity-Time) symmetry via the mass difference measurement of multi-strange baryons (\rmXiM[$dss$] and \mbox{\rmAxiP[$\bar{d}\bar{s}\bar{s}$]}, and \rmOmegaM[$sss$] and \rmAomegaP[$\bar{s}\bar{s}\bar{s}$]) in proton-proton collisions at \sqrtS = 13 \tev. The current mass and mass difference values given by the \textit{Particle Data Group} (PDG) for these two baryons relying on measurements with relatively low statistics, it becomes now possible to improve them in order to test the CPT symmetry to an unprecedented level of precision, thanks to the abundant production and detection of these baryons by ALICE at the LHC. The total uncertainty on the mass values has been reduced by a factor 1.19 for the \rmXiM and \rmAxiP, and 9.26 for the  \rmOmegaM et \rmAomegaP. Concerning the mass differences, their precision has been improved by 20\% for the \rmXi, and by more than a factor two for the \rmOmega.

The second analysis aims to provide a better understanding of the production mechanisms of strange quarks in proton-proton collisions at \sqrtS = 13 \tev. This is achieved by studying the correlations between identified particles. In practice, this analysis focuses specifically on correlations between a multi-strange baryon -- \rmXiPM or \rmOmegaPM\ -- and a \rmPhiMes[$s\bar{s}$] resonance. The first results show no correlation with the rapidity separation while the production of \rmPhiMes increases in the vicinity (in azimuth) of a \rmXiPM in both minimum-bias and high-multiplicity proton-proton collisions. A similar trend can be observed for \rmOmegaPM-\rmPhiMes correlation in high-multiplicity events. The comparison to QCD-inspired Monte Carlo predictions shows that \Pythiaeight overestimates \rmXiPM-\rmPhiMes correlation with the azimuth in minimum-bias proton-proton collisions, while \EposFour underestimates it. This suggests that the correlated production of strange hadrons is likely an interplay between soft and hard hadronisation mechanisms.\\

\noindent\textbf{Key words:} particle physics, heavy-ion physics, ALICE, LHC, CERN, CPT symmetry, correlated production, multi-strange baryons, strange hadrons, strangeness, precision measurement, mass measurement, mass difference measurement.

%Quantum Chromodynamics (QCD), the quantum field theory of the strong force, predicts the existence of an extreme state of nuclear matter in which partons (quarks and gluons) are deconfined and thermalised: this is the so-called \textit{Quark Gluon Plasma} (QGP). Corresponding supposedly to the primordial state of the Universe up to a few micro-seconds after the Big-Bang, the QGP has been studied experimentally at colliders such as the \textit{Large Hadron Collider} (LHC) at CERN in Geneva, during the LHC Run-1 (2009-2013) and Run-2 (2015-2018) data taking periods.
%
%The 5$^{\rm th}$ of July 2022, the LHC has restarted for a third data taking campaign (LHC Run-3), as well as the experiments installed on the ring: ATLAS, CMS, LHCb, -- and the one in which this thesis is carried out -- ALICE (\textit{A Large Ion Collider Experiment}). Dedicated to the study of QCD and QGP, ALICE has been fully revamped and upgraded in order to i) increase the data taking rates and ii) perform more precise measurements. The objective is clear: with the \mbox{LHC Run-3}, ALICE enters into a new age, an era of precision. In that regard, considering the accumulated statistics throughout the LHC Run-2 (about two billions proton-proton collisions at a centre-of-mass energy of 13 \tev), there already exists plenty of precise measurements, especially in the hyperon sector (baryons containing at least one strange quark).
%
%This thesis proposes to analyse -- possibly, one last time -- the data recorded during the LHC Run-2 before moving on to the ones from the LHC Run-3, in order to fully exploit them and push them to their precision limits. To that end, two analyses have been performed.
%
%The first analysis consists in a test of the CPT (Charge-Parity-Time) symmetry via the mass difference measurement of multi-strange baryons (\rmXiM[$dss$] and \rmAxiP[$\bar{d}\bar{s}\bar{s}$], and \rmOmegaM[$sss$] and \rmAomegaP[$\bar{s}\bar{s}\bar{s}$]). The current mass and mass difference values given by the \textit{Particle Data Group} (PDG) for these two baryons relying on measurements with relatively low statistics, it becomes now possible to improve them in order to test the CPT symmetry to an unprecedented level of precision, thanks to the abundant production and detection of these baryons by ALICE at the LHC. The total uncertainty on the mass values has been reduced by a factor 1.19 for the \rmXiM and \rmAxiP, and 9.26 for the  \rmOmegaM et \rmAomegaP. Concerning the mass differences, their precision has been improved by 20\% for the \rmXi, and by more than a factor two for the \rmOmega.
%
%The second analysis aims to provide a better understanding of the production mechanisms of strange quarks in proton-proton collisions at \sqrtS = 13 \tev. This is achieved by studying the correlations between identified particles. In practice, this analysis focuses specifically on correlations between a multi-strange baryon -- \rmXiPM or \rmOmegaPM\ -- and a \rmPhiMes resonance. The first results show no correlation with the rapidity separation while the production of \rmPhiMes increases in the vicinity (in azimuth) of a \rmXiPM in both minimum-bias and high-multiplicity proton-proton collisions. A similar trend can be observed for \rmOmegaPM-\rmPhiMes correlation in high-multiplicity events. The comparison to QCD-inspired Monte Carlo predictions shows that the \Pythiaeight overestimates \rmXiPM-\rmPhiMes correlation with the azimuth in minimum-bias proton-proton collisions, while \EposFour underestimates it. This suggests that the strangeness production mechanism is likely an interplay between soft and hard hadronisation mechanisms.\\
%
%\noindent\textbf{Key words:} particle physics, heavy-ion physics, ALICE, LHC, CERN, CPT symmetry, correlated production, multi-strange baryons, strange hadrons, strangeness, precision measurement, mass measurement, mass difference measurement.



    

