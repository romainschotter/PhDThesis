\chapter*{Résumé}

La chromodynamique quantique (QCD) prédit l’existence d’un état extrême de la matière nucléaire dans lequel les quarks et gluons sont déconfinés et thermalisés : il s'agit du plasma de quarks et de gluons, aussi appelé \textit{Quark-Gluon Plasma} (QGP). Le QGP a fait l’objet d’études auprès de collisionneurs, notamment au LHC au CERN à Genève, au cours des prises de données du LHC Run-1 (2009-2013) et \mbox{Run-2} (2015-2018). Le 5 juillet 2022, le LHC entre à nouveau en fonctionnement pour une troisième campagne de prise de données (LHC Run-3), ainsi que l'expérience dans laquelle s’effectue cette thèse, ALICE. Ce sujet de thèse propose d’examiner -- une dernière fois peut-être -- les données collectées au cours du \mbox{LHC Run-2} avant de passer à celles du LHC Run-3, afin de les exploiter pleinement et les pousser à leurs limites en terme de précision. À cette fin, deux analyses ont été réalisées.

L'analyse principale porte sur le test de la symétrie CPT (Charge-Parité-Temps) via la mesure de la différence de masse de baryons multi-étranges (\rmXiM[$dss$] et \mbox{\rmAxiP[$\bar{d}\bar{s}\bar{s}$]}, et \rmOmegaM[$sss$] et \rmAomegaP[$\bar{s}\bar{s}\bar{s}$]). Les valeurs actuelles de masses et différences de masse du \textit{Particle Data Group} (PDG) pour ces deux baryons s’appuyant sur des mesures de relativement faibles statistiques, il est désormais possible de les améliorer en vue de tester la symétrie CPT avec une précision inégalée, grâce à l’abondante production et détection de ces baryons par ALICE au LHC. L’incertitude totale sur les valeurs de masse se retrouve réduite d’un facteur 1.19 pour les \rmXiM et \rmAxiP,\break et 9.26 pour les particules \rmOmegaM et \rmAomegaP. Quant aux différences masses, leur précision a été améliorée de 20\% pour les \rmXi et de plus d'un facteur deux pour les \rmOmega.
%Celles-ci sont, jusqu’à présent, les mesures les plus précises de la masse des \rmXi et des \rmOmega. 

La seconde analyse vise à mieux comprendre les mécanismes de production des quarks étranges dans les collisions proton-proton à \sqrtS =  13 \tev. Cela passe par l'étude des corrélations entre particules identifiées. En réalité, cette analyse se concentre specifiquement sur les corrélations entre un baryon multi-étrange -- \rmXiPM ou \rmOmegaPM\ -- et une résonance \rmPhiMes[$s\bar{s}$]. Les premiers résultats ne montrent aucune corrélation avec la séparation en rapidité alors que la production des \rmPhiMes augmente, lorsque celles-ci se trouvent à proximité (en azimut) d'un~\rmXiPM dans les collisions proton-proton de biais-minimum et de haute-multiplicité. Une tendance similaire peut être observée pour la corrélation \rmOmegaPM-\rmPhiMes dans les événements de haute multiplicité. La comparaison avec les prédictions Monte Carlo inspirées de la QCD montre que \Pythiaeight surestime la corrélation azimutale dans les collisions proton-proton de biais-minimum, alors que \EposFour la sous-estime. Cela suggère que la production corrélée d'hadrons étranges consiste vraisemblablement en une combinaison de mécanismes d'hadronisation doux et durs.\\

\noindent\textbf{Mots clés:} physique des particules, physique des ions lourds, ALICE, LHC, CERN, symétrie CPT, production corrélée, baryons multi-étranges, hadrons étranges,\break étrangeté, mesure de précision, measure de masse, mesure de différence de masse.
    
%La chromodynamique quantique (QCD, pour \textit{\textbf{Q}uantum \textbf{C}hromo\textbf{D}ynamics}), théorie quantique des champs décrivant l’interaction forte, prédit l’existence d’un état extrême de la matière nucléaire dans lequel les partons (quarks et gluons) sont déconfinés et thermalisés : on le nomme plasma de quarks et de gluons, aussi appelé \textit{Quark-Gluon Plasma} (QGP). Correspondant supposément à l’état de l’Univers quelques microsecondes après le Big Bang, le QGP a fait l’objet d’études auprès de collisionneurs, notamment au \textit{Large Hadron Collider} (LHC) au CERN à Genève, au cours des prises de données du LHC Run-1 (2009-2013) et Run-2 (2015-2018). 
%
%Le 5 juillet 2022, le LHC entre à nouveau en fonctionnement pour une troisième campagne de prise de données (LHC Run-3), ainsi que les expériences installées sur son anneau : ATLAS, CMS, LHCb, -- et celle dans laquelle s’effectue cette thèse -- ALICE (\textit{A Large Ion Collider Experiment}). Cette dernière, dédiée à l’étude de la QCD et du QGP, a profité d’une cure de jouvence ainsi que de nombreuses améliorations dans le but i) d’enregistrer des données à une cadence plus élevée et ii) d’effectuer des mesures plus précises. L’objectif est clair : avec le LHC Run-3, ALICE entre dans un nouvel âge, une ère de précision. À cet égard, au vue de la statistique accumulée au cours du LHC Run-2 (environ deux milliards de collisions proton-proton à une énergie dans le centre de masse de 13 \tev), il existe déjà un certain nombre de mesures extrêmement précise, particulièrement dans le secteur des hypérons (baryons contenant un ou plusieurs quarks étranges).
%
%Ce sujet de thèse propose d’examiner -- une dernière fois peut-être -- les données collectées au cours du LHC Run-2 avant de passer à celles du LHC Run-3, afin de les exploiter pleinement et les pousser à leurs limites en terme de précision. À cette fin, deux analyses ont été réalisées.
%
%La première analyse porte sur le test de la symétrie CPT (Charge-Parité-Temps) via la mesure de la différence de masse de baryons multi-étranges (\rmXiM[$dss$] et \mbox{\rmAxiP[$\bar{d}\bar{s}\bar{s}$]}, et \rmOmegaM[$sss$] et \rmAomegaP[$\bar{s}\bar{s}\bar{s}$]). Les valeurs actuelles de masses et différences de masse du \textit{Particle Data Group} (PDG) pour ces deux baryons s’appuyant sur des mesures de relativement faibles statistiques, il est désormais possible de les améliorer en vue de tester la symétrie CPT avec une précision inégalée, grâce à l’abondante production et détection de ces baryons par ALICE au LHC. L’incertitude totale sur les valeurs de masse se retrouve réduite d’un facteur 1.19 pour les \rmXiM et \rmAxiP,\break et 9.26 pour les particules \rmOmegaM et \rmAomegaP. Quant aux différences masses, leur précision a été améliorée de 20\% pour les \rmXi et de plus d'un facteur deux pour les \rmOmega.
%%Celles-ci sont, jusqu’à présent, les mesures les plus précises de la masse des \rmXi et des \rmOmega. 
%
%La seconde analyse vise à mieux comprendre les mécanismes de production des quarks étranges dans les collisions proton-proton à \sqrtS =  13 \tev. Cela passe par l'étude des corrélations entre particules identifiées. En réalité, cette analyse se concentre specifiquement sur les corrélations entre un baryon multi-étrange -- \rmXiPM ou \rmOmegaPM\ -- et une résonance \rmPhiMes. Les premiers résultats ne montrent aucune corrélation avec la séparation en rapidité alors que la production des \rmPhiMes augmente, lorsque celles-ci se trouvent à proximité (en azimut) d'un~\rmXiPM dans les collisions proton-proton de biais-minimum et de haute-multiplicité. Une tendance similaire peut être observée pour la corrélation \rmOmegaPM-\rmPhiMes dans les événements de haute multiplicité. La comparaison avec les prédictions Monte Carlo inspirées de la QCD montre que \Pythiaeight surestime la corrélation azimutale dans les collisions proton-proton de biais-minimum, alors que \EposFour la sous-estime.\\
%
%\noindent\textbf{Mots clés:} physique des particules, physique des ions lourds, ALICE, LHC, CERN, symétrie CPT, production corrélée, baryons multi-étranges, hadrons étranges,\break étrangeté, mesure de précision, measure de masse, mesure de différence de masse.

%Celles-ci sont motivées par une prédiction de \Pythia (un modèle phénoménologique décrivant la dynamique des collisions proton-proton à haute énergie)
