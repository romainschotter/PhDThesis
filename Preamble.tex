
%_____________________________________________________________________________________
%Pour les documentations sur les differents packages, voir le TeX catalogue on line 
%        http://texcatalogue.sarovar.org/index.html,
%_____________________________________________________________________________________


\usepackage[T1]{fontenc}    
    % for output font rendering, T1 rendering
    % textsc into section title http://en.wikibooks.org/wiki/LaTeX/Fonts#Font_encoding
 \usepackage{lmodern}
%\usepackage[varg]{txfonts}   % Web of Conferences font
    % choose specific font, lmodern (default : if installed = cm-super, rendering ok!)
    % There is nothing to change in your document to use CM Super fonts (assuming they are installed), 
    %      they will get loaded automatically if you use T1 encoding. 
    %      For lmodern, you will need to load the package after the T1 encoding has been set
    
%\usepackage[latin1]{inputenc}
\usepackage[utf8]{inputenc} % for input source code, accented characters like French spelling...
\usepackage{graphicx,subfigure}
%\usepackage{subcaption} %for subfigures, added 10.dec.20

%\usepackage[english,francais]{babel}
\usepackage{amsmath}
\usepackage{amssymb}  % boldsymbol, special characters
\usepackage{mathrsfs} % pour le L du lagrangien ...
\usepackage{MnSymbol} % pour fivedots
\usepackage{bbding}   % pour checkmark spéciaux dans tableau d'inventaire
\usepackage{dictsym}  % pour dsaagricultural du tableau d'inventaire
\usepackage{manfnt}   % pour le signe "virage dangereux" \textlhdbend
%\usepackage{marvosym}  % Special Fonts, if need be. http://texdoc.net/texmf-dist/doc/fonts/marvosym/marvodoc.pdf
% \usepackage{eurosym}  % symbole de l'euro, \officialeuro
%     \DeclareUnicodeCharacter{20AC}{\euro}  % i.e. make the translation € UTF-8 char to be the LaTeX € symbol
\usepackage[retainorgcmds]{IEEEtrantools} % equation à la IEEE
\usepackage{multirow}
\usepackage{xspace} 
\usepackage{lineno}     % for line numbering
%\usepackage[showframe=false]{geometry}
\usepackage{changepage} % for managing locally the width allowed for the text = useful for shifting leftwards too wide tables or figure
                        % e.g. \begin{adjustwidth}{-1cm}{} ... \end{adjustwidth}
\usepackage{threeparttable} % for enabling footnote within a table
\usepackage{pdflscape}  % Rotation of tables (better handling), for pdflatex
\usepackage{dcolumn}  % to align column on the decimal point
\usepackage{enumitem} % to be able to have enumerate a,b... A,B... i,ii...
                        % \begin{enumerate}[label=(\alph*)], [label=(\Alph*)], [label=(\roman*)]
\usepackage[normalem]{ulem} % pour tuner le soulignage, [normalem] = pour préserver le comportement normale de emph
\usepackage{longtable}
\usepackage{float}
\usepackage{cellspace, rotating, tabularx, booktabs}
%\usepackage{garamondx}
\addparagraphcolumntypes{X}

\usepackage{ifthen} % execute code with condition
\usepackage{nicefrac} % to get nice fractions of type x / y

\usepackage{indentfirst}

\usepackage{titlesec} % controls title spacing
\titlespacing{\section}{0pt}{4.5ex plus 1ex minus .2ex}{2.3ex plus .2ex}
\titlespacing{\subsection}{0pt}{4.5ex plus 1ex minus .2ex}{2.3ex plus .2ex}
\titlespacing{\subsubsection}{0pt}{4.5ex plus 1ex minus .2ex}{2.3ex plus .2ex}

\usepackage{feynmp-auto} %to draw Feyman diagram
\usepackage{feynmp}
\DeclareGraphicsRule{*}{mps}{*}{}
%\setlength{\unitlength}{1cm}

% after babel
\usepackage{datetime} % pour afficher l'heure avec la commande \currenttime = \xxivtime

\usepackage[outercaption]{sidecap}
\usepackage{afterpage}
\usepackage{dirtytalk}
\usepackage{textcomp}
\usepackage[export]{adjustbox}% http://ctan.org/pkg/adjustbox
\usepackage{tablefootnote}
\usepackage{diagbox}
\usepackage{cancel}
\usepackage{url}

%%%%%%%%%%%%%package Romain
% \usepackage[lofdepth,lotdepth,caption=false]{subfig}  
    % FIXME pb incompatible with hyperref apparently, enable to compile the document locally
    % https://tex.stackexchange.com/questions/129791/using-subfloat-with-hyperref
%%%%%%%%%%%%%end package Romain


% Package nécessaire pour les remerciements
% \usepackage{endnotes}
% \renewcommand{\notesname}{Notes de remerciements}



%%% OPTION - pdflatex compiler
\usepackage[pdftex,usenames,dvipsnames,table]{xcolor}
%\usepackage{xcolor}
%\usepackage[pdftex]{graphicx}
\usepackage{eso-pic,graphicx, transparent}
\usepackage{epstopdf}
\DeclareGraphicsExtensions{.jpg,.eps,.png,.pdf}


%%% OPTION - Latex compiler
% \usepackage[usenames,dvipsnames]{color}
% \usepackage[dvips]{graphicx}
% \DeclareGraphicsExtensions{.jpg,.eps,.pdf,.png}
% %\DeclareGraphicsRule{.jpg}{eps}{.jpg.bb}{`./jpeg2ps/jpeg2ps -h #1}
% % % Truc de Julien : permet d'inclure des jpg au lieu d'eps ("wrapper" des jpeg en PS Level 2)
% % % Particulièrement utile pour les fichiers images volumineux 
% % % Voir également le shell script Jpg-DoBdngBox.sh et les fichiers *.jpg.bb
% 
% % \DeclareGraphicsRule{.eps.zip}{eps}{.eps.bb}{`unzip -p #1}%   zipped EPS
% % \DeclareGraphicsRule{.eps.gz}{eps}{.eps.bb}{`gunzip -c #1}%   gzipped EPS
% %         \DeclareGraphicsRule{.jpg}{eps}{.jpg.bb}{`convert #1 eps:-}%         JPEG
% %         \DeclareGraphicsRule{.gif}{eps}{gif.bb}{`convert #1 eps:-}%      GIF
% \DeclareGraphicsRule{.png}{eps}{.png.bb}{`convert #1 eps:-}%      PNG
% %         \DeclareGraphicsRule{.tif}{eps}{.bb}{`convert #1 eps:-}%      TIFF
% \DeclareGraphicsRule{.pdf}{eps}{.pdf.bb}{`convert #1 eps:-}%      PDF-graphics



\usepackage[bookmarks,backref=page]{hyperref}
        % NOTE :
        % For colour choices : https://en.wikibooks.org/wiki/LaTeX/Colors

        \makeatletter
        \Hy@AtBeginDocument{%
        \def\@pdfborder{0 0 1}% Overrides border definition set with colorlinks=true
        \def\@pdfborderstyle{/S/S/W 1}% Overrides border style set with colorlinks=true
                                        % Hyperlink border style will be framed of width 1pt
                                        % https://tex.stackexchange.com/questions/26071/how-can-i-have-colored-and-underlined-links-with-hyperref
        }
        \makeatother

        \hypersetup{colorlinks=true, 
        % NOTE:
        %   - to have coloured frames + have coloured links, uncomment l.89+90 above and colorlinks=true
        %   - to have coloured frames + kill coloured links,   comment l.89+90 above and colorlinks=false
        %   - to kill coloured frames + have coloured links,   comment l.89+90 above and colorlinks=true
        % See https://tex.stackexchange.com/questions/50747/options-for-appearance-of-links-in-hyperref
                    linktocpage,
                    citebordercolor=ForestGreen,
                    linkbordercolor=Red,
                    urlbordercolor=Cerulean,
                    % menubordercolor= [rgb 1 0 0]
                    % filebordercolor= [rgb 0 .5 .5]
                    % runbordercolor= [rgb 0 .7 .7]
                    %allbordercolors=Red
                    citecolor=MidnightBlue, 
                    filecolor=MidnightBlue, 
                    linkcolor=MidnightBlue, 
                    urlcolor=MidnightBlue}
        % \hypersetup{colorlinks,  linktocpage, citecolor=Gray, filecolor=Gray, linkcolor=Gray, urlcolor=Gray} %FIXME : N&B
        \urlstyle{sf} % change the url font for sans serif sf, else rm
        
%\usepackage[backref=page]{hyperref}

\renewcommand*{\backref}[1]{}
\renewcommand*{\backrefalt}[4]{\footnotesize[{\footnotesize%
    \ifcase #1 Not cited.%
          \or Cited on page~#2.%
          \else Cited on pages #2.%
    \fi%
}]}       

\usepackage[comma,square,numbers,sort&compress]{natbib}
% \usepackage[numbers, sort&compress]{mynatbib} %pour avoir une génération automatique de réf. bib [1-6] au lieu de lister [1, 2, 3, 4, 5, 6].
% \usepackage{hypernat} 
% L'usage de mynatbib = version locale modifiée de natbib est fait pour 
% empêcher la réinterprétation de la commande \newblock de la bibliographie
% En conséquence, tout ce qui est \newblock dans natbib.sty = changer pour "\\"

\usepackage{array} 
        %pour avoir \extrarowheight = gestion de la hauteur de ligne (voir chap. 4)
        % Ajout de hauteur suppl aux lignes pour éviter que la ligne horizontale ne touche le texte.
%        \setlength{\extrarowheight}{3 pt}  %à mettre au dessus du tableau considérée. 0 pt = valeur par défaut
        %Option m{largeur colonne} permet de centrer le texte verticalement (package array)
        
\usepackage{textpos}  % pour la realisation de la page de titre
        \setlength{\TPHorizModule}{10mm}
        \setlength{\TPVertModule}{10mm}
                        % definition de l'unite de base de longueur pour le placement avec textpos

\usepackage{epigraph} %allow rajout d'epigraphe (voir conclusion pour exemple)
        \setlength{\epigraphwidth}{80mm}
        \renewcommand{\epigraphsize}{\footnotesize}
        
%\usepackage{fancyhdr} %modification des bas de pages et en-tetes
%\usepackage{tocbibind} %allow integration de biblio+index dans la TOC(a comparer avec addcontentsline)
%\usepackage{bibunits} %permet de faire une biblio par partie, chapitre, section...
        

%__________ Redefine style of the header top line, with section names
        
\def\MakeUppercase#1{{ \textsf{\small #1} }}
% MakeUppercase is already defined into LaTeX





%__________Mise en place des \newcommand generales
%\newcommand{\Bluecite}[1]{\textcolor{Blue}{\cite{#1}}}
\newcommand{\BoldSubSection}[1]{\noindent \textbf{\textsl{#1}}\\   \addcontentsline{toc}{subsection}{ \textsl{\textcolor{Gray}{\small #1 }}}  }

\newcommand{\SlantedSubSubSection}[1]{ -- \textsl{#1}\\   \addcontentsline{toc}{subsubsection}{ \textsl{\textcolor{Gray}{\small #1 }}}  }


\newcommand{\urlscpt}[1]{\hbox{\scriptsize \url{#1}}}
        % reduce the size of Internet address

\newcommand{\refmark}[1]{\hbox{\scriptsize $^{\ref{#1}}$}}
        %pour pouvoir faire une référence multiple a une note de bas de page
        % voir exemple dans le chap. 4

\renewcommand\descriptionlabel[1]{\hspace\labelsep\normalfont\itshape #1 :}
%in the environment "description", produce labels in italic, with colon at the end


% write roman numbers in your text in lowercase or uppercase
\newcommand{\upperRomannumeral}[1]{\uppercase\expandafter{\romannumeral#1}}
\newcommand{\lowerromannumeral}[1]{\romannumeral#1\relax}


%\renewcommand{\newblock}{\\}
% pour revenir à la ligne après chaque bloc, dans la bibliographie = réinterpréter newblock en \\

\newenvironment{BulletList}%
{ \begin{list}%
        {$\bullet$}%
        {\setlength{\labelwidth}{30pt}%
         \setlength{\leftmargin}{35pt}%
         \setlength{\itemsep}{\parsep}%
         \setlength{\topsep}{\parsep}}}
{ \end{list} }





%__________Definition of colours
% http://cloford.com/resources/colours/500col.htm

% in gray shades
\definecolor{DarkGray}{RGB}{60,60,60}
\definecolor{LightGray}{RGB}{145,145,145}

% in red shades
\definecolor{Sepia}{RGB}{94,38,18}
\definecolor{IndianRed}{RGB}{176,23,31}
\definecolor{OrangeRed4}{RGB}{139,37,0}
\definecolor{DarkRed}{RGB}{139,0,0}

% in orange shades
\definecolor{Orange2}{RGB}{238,154,0}
\definecolor{Goldenrod1}{RGB}{255,193,37}
\definecolor{Goldenrod2}{RGB}{238,180,34} 

% in blue shades
\definecolor{DarkSlateBlue}{RGB}{72,61,139}
\definecolor{Cobalt}{RGB}{61,89,171}
\definecolor{RoyalBlue4}{RGB}{39,64,139}
\definecolor{DodgerBlue4}{RGB}{16,78,139}
\definecolor{SteelBlue4}{RGB}{54,100,139}
\definecolor{DeepSkyBlue4}{RGB}{0,104,139}

% in green shades
\definecolor{LightGreen}{RGB}{0,200,0}





%__________Insertion of source codes


\usepackage{listings}
% In order to include source code from various prog language
% For documentation : 
%   http://en.wikibooks.org/wiki/LaTeX/Source_Code_Listings
%   http://www.ctan.org/tex-archive/macros/latex/contrib/listings/


\lstset{ %
  backgroundcolor=\color{white},          % choose the background color; you must add \usepackage{color} or \usepackage{xcolor}
  basicstyle=\footnotesize\ttfamily,      % the font and size that are used for the code
  breakatwhitespace=false,                % sets if automatic breaks should only happen at whitespace
  breaklines=true,                        % sets automatic line breaking
  captionpos=t,                           % sets the caption-position to top
  commentstyle=\color{LightGray}\upshape, % comment style
  deletekeywords={...},                   % if you want to delete keywords from the given language
  %escapeinside={\%*}{*)},                % if you want to add LaTeX within your code
  extendedchars=true,                     % lets you use non-ASCII characters; for 8-bits encodings only, does not work with UTF-8
  frame=tlbr,                             % adds a frame around the code : single, t b l r, T B L R
  keepspaces=false,                        % keeps spaces in text, useful for keeping indentation of code (possibly needs columns=flexible)
  keywordstyle=\bfseries\color{black},    % keyword style
  identifierstyle=,
  language=C,                        % the default language of the code
  %morekeywords={*,...},                  % if you want to add more keywords to the set
  numbers=left,                           % where to put the line-numbers; possible values are (none, left, right)
  numbersep=8pt,                          % how far the line-numbers are from the code
  numberstyle=\tiny\color{LightGray},     % the style that is used for the line-numbers
  rulecolor=\color{LightGray},            % if not set, the frame-color may be changed on line-breaks within not-black text (e.g. comments (green here))
  showspaces=false,                       % show spaces everywhere adding particular underscores; it overrides 'showstringspaces'
  showstringspaces=false,                 % underline spaces within strings only
  showtabs=false,                         % show tabs within strings adding particular underscores
  stepnumber=1,                           % the step between two line-numbers. If it's 1, each line will be numbered
  stringstyle=\color{Goldenrod2},         % string literal style
  tabsize=2,                              % sets default tabsize to 2 spaces
  caption=\lstname,                       % show the filename of files included with \lstinputlisting; also try caption instead of title
  xleftmargin=10pt,                       % size of the left margin
  belowcaptionskip=1.2\baselineskip,
  aboveskip=1\baselineskip,               % vertical skip above the listings envt
  belowskip=1\baselineskip,
}

\lstdefinestyle{customC++}{
  language=C++
}

\lstdefinestyle{customBash}{
  language=bash
}

\lstdefinestyle{customCmnd}{
  language=sh,
  identifierstyle=\color{blue},
  morecomment=[l][\color{LightGray}]{!\ } % define the rest of the whole line as comments
}

% look-up table to have the listings package fully compatible with UTF-8 extended char (extendedchars has to be true).
\lstset{literate=
  {á}{{\'a}}1 {é}{{\'e}}1 {í}{{\'i}}1 {ó}{{\'o}}1 {ú}{{\'u}}1
  {Á}{{\'A}}1 {É}{{\'E}}1 {Í}{{\'I}}1 {Ó}{{\'O}}1 {Ú}{{\'U}}1
  {à}{{\`a}}1 {è}{{\'e}}1 {ì}{{\`i}}1 {ò}{{\`o}}1 {ò}{{\`u}}1
  {À}{{\`A}}1 {È}{{\'E}}1 {Ì}{{\`I}}1 {Ò}{{\`O}}1 {Ò}{{\`U}}1
  {ä}{{\"a}}1 {ë}{{\"e}}1 {ï}{{\"i}}1 {ö}{{\"o}}1 {ü}{{\"u}}1
  {Ä}{{\"A}}1 {Ë}{{\"E}}1 {Ï}{{\"I}}1 {Ö}{{\"O}}1 {Ü}{{\"U}}1
  {â}{{\^a}}1 {ê}{{\^e}}1 {î}{{\^i}}1 {ô}{{\^o}}1 {û}{{\^u}}1
  {Â}{{\^A}}1 {Ê}{{\^E}}1 {Î}{{\^I}}1 {Ô}{{\^O}}1 {Û}{{\^U}}1
  {œ}{{\oe}}1 {Œ}{{\OE}}1 {æ}{{\ae}}1 {Æ}{{\AE}}1 {ß}{{\ss}}1
  {ç}{{\c c}}1 {Ç}{{\c C}}1 {ø}{{\o}}1 {å}{{\r a}}1 {Å}{{\r A}}1
  {€}{{\EUR}}1 {£}{{\pounds}}1 
}






%__________Mise en place de la structure en chap., section ... + toc

\renewcommand{\thepart} {\Alph{part}.}
\renewcommand{\thechapter} {\arabic{chapter}}
\renewcommand{\thesection} {\thechapter|{\small \Roman{section}}}
\renewcommand{\thesubsection}   {\thesection-{\small \Alph{subsection}}}
\renewcommand{\thesubsubsection} {\thesection-{\small \Alph{subsection}}.{\footnotesize \roman{subsubsection}}}
% pour avoir une structure du type "I.A -1.i" au lieu de "1.1.1.1"


% TOC normal
\setcounter{tocdepth}{3}     % Arret au niveau des subsubsection
\setcounter{secnumdepth}{3}  % Arret de la numerotation au niveau des subsubsections


% - 1.
\usepackage{titlesec}
% definition utilisateur du style des chap., sections, ...

        \usepackage{etoolbox}
        % Bug in sectionning : plus de numéros apparent dans le texte
        % Fix = http://tex.stackexchange.com/questions/299969/titlesec-loss-of-section-numbering-with-the-new-update-2016-03-15
        \makeatletter
        \patchcmd{\ttlh@hang}{\parindent\z@}{\parindent\z@\leavevmode}{}{}
        \patchcmd{\ttlh@hang}{\noindent}{}{}{}
        \makeatother


%font style = \sffamily, \ttfamily, \rmfamily
%font series = \bfseries, \mdseries
%font shape = \upshape, \itshape, \scshape, \slshape
% http://www.math.jussieu.fr/~goutet/latex/seance_6/seance_6.pdf

%\titleclass{\part}{straight}
\titleformat{\part}[display]
  {\normalfont\sffamily\huge\bfseries\color{OrangeRed4}}
  % FIXME {\normalfont\sffamily\huge\bfseries\color{Black}}
  {-- \partname\ \thepart}{20pt}{\Huge}

% \titleformat{\chapter}[display]
%   {\normalfont\sffamily\huge\bfseries\color{Sepia}}
%   % FIXME {\normalfont\sffamily\huge\bfseries\color{Black}}
%   {-- \chaptertitlename\ \thechapter~-- }{20pt}{\Huge}

% Title format: chapter
\titleformat{\chapter}[display]
	{\fontsize{30}{20}\selectfont\bfseries\filright}
	{\chaptertitlename}
	{20pt}
	{
	\ifthenelse{\thechapter>0}
	{\fontsize{60}{0}\selectfont\arabic{chapter}\hspace{0pt}\hspace{2mm}{|   }\hspace{0pt}}
	{\hspace{0pt}}
	\fontsize{30}{60}\bfseries 
	}
  
%\titleclass{\section}{straight}  
\titleformat{\section}[hang]
%  {\normalfont\rmfamily\Large\bfseries\color{OrangeRed4}}
  {\normalfont\rmfamily\Large\bfseries\color{RoyalBlue4}}
  % FIXME {\normalfont\rmfamily\Large\bfseries\color{Black}}
  {\Roman{section}}{1em}{}

\titleformat{\subsection}[hang]
  {\normalfont\rmfamily\large\bfseries\color{DarkGray}}
  % FIXME {\normalfont\rmfamily\large\bfseries\color{Black}}
  {~~\Roman{section}-{\small \Alph{subsection}}}{1em}{}

\titleformat{\subsubsection}[hang]
  {\normalfont\rmfamily\large\bfseries\color{DarkGray}}
  % FIXME {\itshape\rmfamily\large\bfseries\color{Black}}
  {~~~~\Roman{section}-{\small \Alph{subsection}}.{\footnotesize \roman{subsubsection}}}{1em}{}

% - 2.
\usepackage{titletoc}
% definition utilisateur du style du TOC ...

\titlecontents{part}%
[2em]% retrait à gauche
{\addvspace{5em plus 0pt}\flushright\bfseries\color{MidnightBlue}}% matériel avant commun aux entrées numérotées ou pas
{\contentslabel{2.0em}}% avant lorsqu'il y a un numéro
{\hspace{-2.0em}}% avant lorsqu'il n'y a pas de numéro
{}% points de suspension et numéro de page
[\addvspace{2em}]% matériel après



\titlecontents{chapter}%
[2.5em]% retrait à gauche
{\addvspace{3em plus 0pt}\bfseries}% matériel avant commun aux entrées numérotées ou pas
{\contentslabel{2.5em}}% avant lorsqu'il y a un numéro
{\hspace{-2.5em}}% avant lorsqu'il n'y a pas de numéro
{\dotfill\contentspage}% points de suspension et numéro de page
[\addvspace{0pt}]% matériel après


\titlecontents{section}%
[4.5em]% retrait à gauche
%{\addvspace{8pt}\mdseries\color{OrangeRed4}}% matériel avant commun aux entrées numérotées ou pas
{\addvspace{8pt}\mdseries\color{RoyalBlue4}}% matériel avant commun aux entrées numérotées ou pas
{\contentslabel{3.5em}}% avant lorsqu'il y a un numéro
{\hspace{-3.5em}}% avant lorsqu'il n'y a pas de numéro
{\dotfill\contentspage}% points de suspension et numéro de page
[\addvspace{0pt}]% matériel après


\titlecontents{subsection}%
[5.5em]% retrait à gauche
{\mdseries\color{DarkGray}}% matériel avant commun aux entrées numérotées ou pas
{\contentslabel{4.5em}}% avant lorsqu'il y a un numéro
{\hspace{-4.5em}}% avant lorsqu'il n'y a pas de numéro
{\dotfill\contentspage}% points de suspension et numéro de page
[\addvspace{-0pt}]% matériel après


\titlecontents{subsubsection}%
[6.5em]% retrait à gauche
{\mdseries\mdseries\color{DarkGray}}% matériel avant commun aux entrées numérotées ou pas
{\contentslabel{5.5em}}% avant lorsqu'il y a un numéro
{\hspace{-5.5em}}% avant lorsqu'il n'y a pas de numéro
{\dotfill\contentspage}% points de suspension et numéro de page
[\addvspace{-0pt}]% matériel après


% - 3 : Corrections nécessaires pour les TOC partiels
\makeatletter
\AtBeginDocument{%
    \def\ttl@gobblecontents#1#2#3#4{\ignorespaces}%
}
\makeatother

% - 4 : Correction nécessaire pour améliorer la référence à une (sous-sous-)section

%\newcommand{\refSection}[1]{\mbox{\kern-0.1em \ref{#1}}}
\newcommand{\refSubSection}[1]{\mbox{\kern-0.6em \ref{#1}}}
\newcommand{\refSubSubSection}[1]{\mbox{\kern-1.2em \ref{#1}}}
        % suite à la redéfinition du style de la hiérarchie (II.A.1.i)
        % faire référence à un paragraphe laisse beaucoup d'espace devant la référence : "Dans le paragaphe      I.C.2.i"













%__________Option de draft : Définition de la version


\newcommand{\version}[2]{ [Version {#1} - {\scriptsize (git rev.{#2})} -  \today, \currenttime] }


% _____ Option 1 - Simple
% to get a light mark in the diagonal of every page
% drawback :    for people commenting on the pdf, the diagonal can mess up 
%               the selection of words you would like to comment on.
%               Said to be inconvenient.
% \usepackage{draftwatermark}
% \SetWatermarkLightness{0.90}
% \SetWatermarkAngle{90}
% \SetWatermarkScale{0.3}
% \SetWatermarkText{\today, \currenttime}


% _____ Option 2 - More complex : to get the draftwatermark in the right margin = https://ctan.org/pkg/background
\usepackage[color=gray]{background}
\backgroundsetup{
    position={+8.7cm,-6cm},
    firstpage=true, % does not work here apparently
    angle=-90,
    opacity=0.5,
    scale=2,
   %contents=Draft % the exact text is set in the master document
}







%__________Option de draft : notes en marge

\usepackage[colorinlistoftodos]{todonotes} % disable, obeyDraft, obeyFinal




\pagestyle{headings}


